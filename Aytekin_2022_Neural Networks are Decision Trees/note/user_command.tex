% addFigure のデフォルト幅
\newlength{\defaultFigureWidth}
\setlength{\defaultFigureWidth}{0.5\linewidth}
% \addFigure[width]{図ファイル名}{キャプション}{ラベル}
\newcommand{\addFigure}[4][\the\defaultFigureWidth]{
    \begin{figure}[hb]
        \centering
        \includegraphics[width=#1]{#2}
        \caption{#3}
        \label{fig:#4}
    \end{figure}
}

% 図<label名> を参照
\newcommand{\figref}[1]{図~\ref{fig:#1}}

% KLダイバージェンス
% || で区切る
\DeclareMathOperator{\KL}{KL}
\DeclarePairedDelimiterX{\divbrace}[2]{[}{]}{%
    #1\;\delimsize\|\;#2%
}
\newcommand{\KLdiv}{
    \KL\divbrace
}

\usepackage{fancyvrb}