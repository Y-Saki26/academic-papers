\usepackage[dvipdfmx,hidelinks]{hyperref} %赤い bounding box を消すため hidelinks を指定
\usepackage{pxjahyper}% hyperrefの日本語対応
\usepackage[dvipdfmx]{graphicx}
\usepackage{tikz}
\usepackage{here} % 図表をその場に出力する[H]オプションの追加
\usepackage{amsmath,amssymb}
\usepackage{mathtools}
\usepackage{siunitx}
\usepackage[version=4]{mhchem}
\usepackage{physics} % `\qty`の名前衝突のためsiunitxのあとに読み込む
\usepackage{url}
\usepackage{comment}
\usepackage{bm}

\usepackage{cprotect} % box内で\verbなどを使う
\usepackage{moreverb} % verbatimの拡張
% 定義リスト
% http://texuttex.g2.xrea.com/pdf/Entwurf02.pdf p89
\newdimen\hangindentlength
\hangindentlength=4zw
  \newcommand{\deflist}[2]{%
  \noindent
  \hangindent=\hangindentlength
  \textbf{#1}\\
  #2\par
}

% align, gather 環境でページをまたがせる
\allowdisplaybreaks[4]

% 基本的な括弧類の定義
% https://mathlandscape.com/declarepaireddelimiter/#toc3
%\DeclarePairedDelimiter{\abs}{\lvert}{\rvert} % | | absolute value
%\DeclarePairedDelimiter{\norm}{\lVert}{\rVert} % || || norm
\DeclarePairedDelimiter{\rbra}{\lparen}{\rparen} % () round brackets
\DeclarePairedDelimiter{\cbra}{\lbrace}{\rbrace} % {} curly brackets
\DeclarePairedDelimiter{\sbra}{\lbrack}{\rbrack} % [] square brackets
\DeclarePairedDelimiter{\abra}{\langle}{\rangle} % < > angle brackets
\DeclarePairedDelimiter{\floor}{\lfloor}{\rfloor} % floor function
\DeclarePairedDelimiter{\ceil}{\lceil}{\rceil} % ceil function


% addFigure のデフォルト幅
\newlength{\defaultFigureWidth}
\setlength{\defaultFigureWidth}{0.5\linewidth}
% \addFigure[width]{図ファイル名}{キャプション}{ラベル}
\newcommand{\addFigure}[4][\the\defaultFigureWidth]{
    \begin{figure}[hb]
        \centering
        \includegraphics[width=#1]{#2}
        \caption{#3}
        \label{fig:#4}
    \end{figure}
}

% 図<label名> を参照
\newcommand{\figref}[1]{図~\ref{fig:#1}}

% KLダイバージェンス
% || で区切る
\DeclareMathOperator{\KL}{KL}
\DeclarePairedDelimiterX{\divbrace}[2]{[}{]}{%
    #1\;\delimsize\|\;#2%
}
\newcommand{\KLdiv}{
    \KL\divbrace
}

\usepackage{fancyvrb}